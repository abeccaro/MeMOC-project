\section{Integer Linear Programming Model}
	In this section will be presented the model implemented using Cplex as solver.
	\subsection{Network Flow Model Representation}
		The problem can be represented on a complete weighted graph $G=(N, A)$ where $N$ is the set of nodes (holes on the board) and $A$ is the set of the arcs $(i, j), ∀ i, j ∈ N$ (trajectory of the drill moving from hole $i$ to hole $j$).To each arc will be associated a weight $c_{ij}$ that represents the time needed to move from starting hole to destination.
		
		Using this representation the problem is equivalent to determine the minimum weight hamiltonian cycle on G, so it is like the very popular Travelling Salesman Problem (TSP).
		
		To solve the TSP problem we can formulate it as a network flow model on G. Given a starting node $0 ∈ N$, let $|N|$ be the amount of its output flow, the problem can be solved by finding the path for which:
		\begin{itemize}
			\item Each node receives 1 unit of flow
			\item Each node is visited once
			\item The sum of costs in selected arcs is minimum
		\end{itemize}
		
	\subsection{Variables and Constraints}
		The problem can then be represented like this:\\
		\newline
		\textbf{SETS:}\\
		\begin{addmargin}[2em]{0em}
			$N$ = the graph nodes (holes)\\
			$A$ = arcs in the form $(i, j), ∀ i, j ∈ N$ (trajectories between holes)\\
		\end{addmargin}
		\textbf{PARAMETERS:}\\
		\begin{addmargin}[2em]{0em}
			$c_{ij}$ = time taken by the drill to move from $i$ to $j$, $∀ (i, j) ∈ A$\\
			$0$ = starting hole, $0 ∈ N$\\
		\end{addmargin}
		\textbf{DECISION VARIABLES:}\\
		\begin{addmargin}[2em]{0em}
			$x_{ij}$ = amount of the flow shipped from $i$ to $j$, $∀ (i, j) ∈ A$\\
			$y_{ij}$ = 1 if arc $(i, j)$ ships some flow, 0 otherwise, $∀ (i, j) ∈ A$\\
		\end{addmargin}
		\textbf{OBJECTIVE FUNCTION:}\\
		\[min \sum_{i, j | (i, j) ∈ A} c_{ij} \cdot y_{ij}\]\\
		\textbf{CONSTRAINTS:}
		\begin{align*}
			&\sum_{j | (0,j) ∈ A} x_{0j} = |N|\\
			\\
			&\sum_{i | (i, k) ∈ A} x_{ik} - \sum_{j | (k, j) ∈ A} x_{kj} = 1 & ∀ k ∈ N \setminus \{0\}\\
			\\
			&\sum_{j | (i, j) ∈ A} y_{ij} = 1 & ∀ i ∈ N\\
			\\
			&\sum_{i | (i, j) ∈ A} y_{ij} = 1 & ∀ j ∈ N\\
			\\
			&x_{ij} \leq |N| \cdot y_{ij} & ∀ (i, j) ∈ A\\
			\\
			&x_{ij} ∈ \mathbb{Z}_{+} & ∀ (i, j) ∈ A\\
			\\
			&y_{ij} ∈ \{0, 1\} & ∀ (i, j) ∈ A
		\end{align*}

		
		